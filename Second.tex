\documentclass[12pt]{article}    % Def du style de document, voir 2.1
\usepackage[francais]{babel} 
\usepackage{times} 
\usepackage{graphicx} 
\usepackage[a4paper,margin=2.5cm]{geometry}              % 2.1.2
\newcommand{\largtt}[1]{{\large\texttt{#1}}}             % 2.18
\begin{document}                   % Debut du texte      % 2.1
\parskip 5pt                                             % 2.4.1

\begin{center}
\LARGE\bfseries Exemple de fichier .tex                  % 2.6
\end{center}

\section{G\'en\'eralit\'es}                              % 2.1

Voici un exemple simple de fichier .tex qui sera interpr\'et\'e
par la commande \largtt{latex} et qui pourra \^etre visualis\'e sur 
\'ecran X par la commande \largtt{xdvi}\footnote{xdvi : {\bfseries x} 
{\bfseries d}e{\bfseries v}ice {\bfseries i}ndependent}.                   

\section{Formules math\'ematiques}                       % 2.7   

Si on encadre une formule par 2 dollars on obtient dans le texte,
$\sum_{i=1}^{n} x_{i} = \int_{0}^{1} f$ alors que si on la met dans
l'environnement \texttt{displaymath} : 
\begin{displaymath}
\sum_{i=1}^{n} x_{i} = \int_{0}^{1} f
\end{displaymath}

\section{Listes}                                         % 2.8

\'Enum\'eration des \'etapes pour la sortie d'un fichier .tex : 
\begin{enumerate}
\item cr\'eation de Fn.tex sous \'editeur,
\item interpr\'etation par la commande {\bfseries latex}, 
      et cr\'eation de Fn.dvi,
\item cr\'eation du fichier imprimable par {\bfseries dvips} puis impression.
\end{enumerate}

\section{Tableaux}                                       % 2.9

\begin{center}
\begin{tabular}{|l|c||r|} \hline                                
ligne1 champ1 & champ2  & 23   \\ \hline
champ1        & champ2  & 123  \\ \hline
ligne3 champ1 & champ2  & 1    \\ \hline
\end{tabular}
\end{center}

\section{Insertion d'une image PostScript}               % 2.10 et 2.11.2

\begin{figure}[h]
\centerline{\includegraphics[width=3cm]{lcomp.eps}}
\caption{Ho !! la belle image.}
\end{figure}

\end{document}               % Fin du texte et du document, voir 2.1
